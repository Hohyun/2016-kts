% 2016 KTS
%

\documentclass{beamer}

\usefonttheme[onlymath]{serif}
\usepackage[hangul]{kotex}
\usepackage{mathtools,booktabs}
\usepackage{fancyvrb}
\usepackage{hyperref}

%\input cfrac
\makeatletter
\def\ltcfrac[#1;#2]{\cfr@c#1,#2,\end}
\def\cfr@c#1,#2\end{\ifx#1\ldots\ddots\else#1\fi
  \ifx#2\end\else+{\strut1\hfill\over\displaystyle\cfr@c#2\end}\fi}
\makeatother

\directlua{dofile("puzzle.lua")}

\newcommand{\alphametics}[2][v]{\directlua{puzzle('#1','#2')}}

%
\title{루아텍 활용}
\subtitle{루아 프로그래밍 중심으로}
\author{남수진}
\date{ 2016년 1월 30일 (토)\\
  2016 한국텍학회 학술대회 및 정기총회 \\
  {\small 고려대학교 하나과학관 206호}}

%
\begin{document}

% 제목
\begin{frame}
  \titlepage
\end{frame}

% 차례
\begin{frame}[t]
  \frametitle{차례}
  \tableofcontents
\end{frame}

%
\section{루아 프로그래밍 언어}

\begin{frame}
  \huge
  \centering 루아 프로그래밍 언어
\end{frame}

\begin{frame}
  \frametitle{Lua/Luajit}
  \begin{itemize}
  \item 루아 프로그래밍 언어 http://kko.to/vdQic7FLO
  \item 루아는 확장 언어와 스크립트 언어를 지향
  \item 충분히 작기 때문에 많은 플랫폼에서 사용할 수 있다.
  \item 적은 수의 기본 데이터형만을 지원
  \item 배열, 집합, 해시 테이블, 리스트, 레코드와 같은 전형적인
    데이터 구조는 모두 연관 배열과 유사한 루아의 테이블 자료형으로 구현
  \end{itemize}
\end{frame}


%
\section{루아텍 소개}

\begin{frame}
  \huge
  \centering 루아텍 소개
\end{frame}

\begin{frame}[fragile]
  \frametitle{\TeX\ 안의 Lua}
  \begin{itemize}
  \item \texttt{luatexref-t.pdf}
  \item \texttt{lualatex-doc.pdf}
  \end{itemize}
\begin{verbatim}
\directlua {<lua code>}
\directlua name {<name>} {<lua code>}
\directlua <number> {<lua code>}
\end{verbatim}
\end{frame}

\begin{frame}[fragile]
  \frametitle{\TeX\ 안의 Lua}
\begin{verbatim}
the standard approximation 
$\pi = \directlua{tex.sprint(math.pi)}$
\end{verbatim}
\begin{center}
  the standard approximation $\pi = \directlua{tex.sprint(math.pi)}$
\end{center}
\end{frame}

\begin{frame}
  \frametitle{\texttt{\string\directlua}}
\end{frame}

\begin{frame}
  \frametitle{luacode 패키지}
  \begin{itemize}
  \item \texttt{\string\luaexec}
  \item \texttt{luacode(*)} 환경
  \end{itemize}
\end{frame}

%
\section{루아텍 예제}

\begin{frame}
  \huge
  \centering 루아텍 예제
\end{frame}


%
\subsection{연분수 순환 주기}

\begin{frame}
  \huge
  \centering 연분수 순환 주기
\end{frame}

\begin{frame}
  \frametitle{연분수 조판}
  \begin{itemize}
  \item 연분수 자동 조판 \url{http://kko.to/vdP6vr99m}
  \item 연분수 조판 소개 \url{http://kko.to/vdPpkDab6}
  \item 연분수 조판 2탄 \url{http://kko.to/vdPkUFe7u}
  \item 연분수 조판 시리즈 마지막 \url{http://kko.to/vdPu5q8AE}
  \end{itemize}
\end{frame}

\begin{frame}
  \frametitle{연분수 순환 주기}
  \[ \sqrt{13}=\ltcfrac[3;1,1,1,1,6,\ldots] =[3;\overline{1,1,1,1,6}] \]
     {\footnotesize
       \url{http://mathworld.wolfram.com/PeriodicContinuedFraction.html}}
\end{frame}

% pcf.lua 코드 시연
% wolfram alpha와 비교
% https://www.codecogs.com/latex/eqneditor.php

%
\subsection{복면산}

\begin{frame}
  \huge
  \begin{center}
    복면산\\
    (Alphametics)
  \end{center}
\end{frame}

\begin{frame}
  \frametitle{복면산: Alphametics}
  \begin{itemize}
  \item A. H. Hunter
  \item H. E. Dudeney
  \end{itemize}
      {\Huge
        \[
        \def\arraystretch{0.7}
        \begin{array}{c@{\,}r}
          & \texttt{SEND} \\
          + & \texttt{MORE} \\
          \cmidrule(lr){1-2}
          & \texttt{MONEY}
        \end{array}
        \]
      }
\end{frame}

\begin{frame}
  \frametitle{복면산: Alphametics}
             {\Huge
               \def\arraystretch{0.7}
               \alphametics[v]{SEND+MORE = MONEY}}
\end{frame}


\begin{frame}
  \huge
  \centering 감사합니다.
\end{frame}

\end{document}

