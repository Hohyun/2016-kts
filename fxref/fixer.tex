%

\documentclass[a4paper]{article}

\usepackage{luacode}
\usepackage{booktabs}
\usepackage[yyyymmdd]{datetime}

\begin{luacode*}
  local json = require "lunajson"
  local http = require "socket.http"
  local currency = require "currency"
  print = tex.print
  
  function fixer (date, base)
     local date = date or "latest"
     local base = base or "EUR"
     local r, c, h = http.request("http://api.fixer.io/"..date.."?base="..base)
     if not r or c ~= 200 then
        return nil, c
     end
     return json.decode(r)
  end

  function country (a, nocur)
     local nocur = nocur or false
     local con = currency[a] or a
     if nocur then
        con = string.gsub(con, "(.+)%s%a+$", "%1")
     end
     return con
  end
\end{luacode*}


\renewcommand{\dateseparator}{-}
\renewcommand{\arraystretch}{1.2}
\newcommand\country[1]{\luaexec{print(country(#1))}}


\begin{document}

%%%
%% 이 부분을 수정. 
\newcommand\mydate{\today} % \today 또는 1999년 이후 "yyyy-mm-dd" 
\newcommand\code{\luastring{USD}} % ISO 4217 code: KRW, EUR, USD, ...
%%%

\thispagestyle{empty}
\begin{center}
  {\Large \country{\code, true} Foreign Exchange Refernce Rates}\\
  \medskip
  \mydate\\
  \bigskip
  \begin{tabular}{ l l r }
    \toprule
    \textbf{\sffamily Code} & \textbf{\sffamily Currency}
    & \textbf{\sffamily Spot}\\
    \midrule
    \luaexec{
      local fx = fixer(\luastring{\mydate}, \code)
      for k, v in pairs(fx.rates) do
         print("{\\sffamily "..k.."} & "..country(k).." & "..v.."\\\\")
      end
    }
    \bottomrule
  \end{tabular}
\end{center}

\end{document}

