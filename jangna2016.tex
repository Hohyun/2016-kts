% 2016 KTS
%

\documentclass{beamer}

\usetheme{metropolis}
\metroset{inner/sectionpage=none}
\metroset{outer/progressbar=head}
\metroset{color/block=fill}
\usefonttheme[onlymath]{serif}

\usepackage{luatexko}
\usepackage{luamplib}
\usepackage{mathtools}
\usepackage{booktabs}
\usepackage{hologo}
\usepackage{array}
\usepackage{luacode}
\usepackage{pgfplots}

\setsanshangulfont{HANDotum-LVT}[
  InterLatinCJK=.125em,
  Expansion, Protrusion,
]


\pgfplotsset{compat=1.5} % to avoid warning

\hypersetup{pdfencoding=auto}

\newcommand{\luatex}{\hologo{LuaTeX}}
\newcommand{\luatexko}{\luatex-\textsf{k}\kern-.0625em\textit{o}}

% title
\title{\luatex\ 활용}
\subtitle{프로그래밍 언어와 조판 시스템의 콜라보}
\author{남수진}
\date{2016년 1월 30일}
\institute{
  2016 한국텍학회 학술대회 및 정기총회 \\
  고려대학교 하나과학관 B206호}
%\titlegraphic{\hfill\includegraphics[height=2cm]{KTSmeta}}

%%
\begin{document}

\maketitle

% 차례
\makeatletter
\patchcmd{\beamer@sectionintoc}{\vskip1.5em}{\vskip0.5em}{}{}
\makeatother

\begin{frame}{차례}
  \setbeamertemplate{section in toc}[sections numbered]
  \tableofcontents
\end{frame}

%%
\section{Lua 프로그래밍 언어}

%
\begin{frame}{Lua 프로그래밍 언어}
  \begin{itemize}
  \item \href{http://www.lua.org}{Lua}, 포루투갈어로 달(Moon)
  \item 1993년, 브라질 PUC-RIO 대학 Tecgraf 연구소 멤버들에 의해서 개발
  \item 기존 응용프로그램을 확장하는 목적
    
    \begin{quote}
      \small
      Lua was originally designed in 1993 as a language for
      \alert{extending software applications} to meet the increasing demand
      for customization at the time.
    \end{quote}
    
  \item 확장성, 단순함, 효율성, 이식성, 개발 편의성에 중점
  \item 현재 버전 5.3.2
  \end{itemize}
\end{frame}

%%
\section{\luatex}

%
\begin{frame}{\luatex}
  \begin{itemize}
  \item \luatex 은 Lua 언어를 내장한 \hologo{pdfTeX}의 확장판이다.
  \item \TeX\ 고유의 기능을 유지하고, Lua를 통하여 \TeX\ 내부의 깊은 부분, 엔진도
    건드릴 수 있다.
  \item 다양한 Lua 라이브러리를 제공,
    \href{http://ftp.ktug.org/tex-archive/systems/doc/luatex/luatexref-t.pdf}%
    {\luatex\ Reference}
  %\item \texttt{mplib} 같은 새로운 기능도 추가.
  \item 현재 버전, {\texttt{\small beta-0.80.0 (TeX Live 2015) (rev 5238)}}
  \end{itemize}
\end{frame}

%
\begin{frame}{\luatex}

  \begin{alertblock}{\ \luatex 의 전형적인 사용례}
    \begin{itemize}
    \item \href{https://github.com/dohyunkim/luatexko}{\luatexko}
    \item \href{https://github.com/lualatex/luamplib}{luamplib 패키지}
    \end{itemize}
  \end{alertblock}
  \begin{exampleblock}{\ \luatex\ 관련 필독서}
    \begin{itemize}
    \item \href{http://ftp.ktug.org/tex-archive/documentation/luatex/lualatex-doc/lualatex-doc.pdf}%
      {A guide to \hologo{LuaLaTeX}}
    \item \href{http://ftp.ktug.org/tex-archive/macros/luatex/latex/luacode/luacode.pdf}%
      {The {\small\texttt{luacode}} package}
    \item \href{http://ftp.ktug.org/tex-archive/macros/luatex/generic/luatexko/luatexko-doc.pdf}%
      {\luatexko\ 간단 메뉴얼}
    \end{itemize}
  \end{exampleblock}
\end{frame}

%
\begin{frame}[fragile]{\luatex}
  \begin{itemize}
  \item \verb+\directlua{<lua code>}+
  \item \verb+\usepackage{luacode}+
    \begin{itemize}
    \item \verb+\luadirect+
    \item \verb+\luaexec+
    \item \texttt{luacode(*)} 환경
    \end{itemize}
  \end{itemize}
  \begin{exampleblock}{\ 입력}
    \verb+the standard approximation+\\
    \verb+$\pi=\directlua{tex.sprint(math.pi)}$+
  \end{exampleblock}
  \begin{alertblock}{\ 결과}
    the standard approximation
    $\pi = \directlua{tex.sprint(math.pi)}$
  \end{alertblock}
\end{frame}

%
\begin{frame}{\luatex}
  Lua가 프로그래밍 언어라는 사실을 상기하면,
  \TeX 만으로는 불가능했거나 어려웠던, 
  \begin{itemize}
  \item 데이타베이스와의 연동 
  \item 네트웍 관련 프로그래밍
  \item 복잡한 수학 계산
  \end{itemize}
  등 컴퓨터 프로그래밍을 동반한 조판이 가능하다.
\end{frame}

%%
\section{예제}

\subsection{오늘의 환율}

%
\begin{frame}{오늘의 환율}
  \begin{center}
    \Large \href{http://fixer.io}{\texttt{Fixer.io}}\\
      {JSON API for foreign exchange rates and\\ 
        currency conversion}
  \end{center}
\end{frame}

%%
\subsection{Periodic Continued Fraction}

% continued fraction
\makeatletter
\def\ltcfrac[#1;#2]{\cfr@c#1,#2,\ldots,\end}
\def\cfr@c#1,#2\end{\ifx#1\ldots\ddots\else#1\fi
  \ifx#2\end\else+{\strut1\hfill\over\displaystyle\cfr@c#2\end}\fi}
\makeatother

%
\begin{frame}{Periodic continued fraction}
  \centering\href{http://kko.to/viBhWcdig}{Wofram MathWorld: Periodic continued fraction}
  \begin{align*}
    \sqrt{2}&=\ltcfrac[1;2,2]=[1;\overline{2}] \\
    \sqrt{3}&=\ltcfrac[1;1,2]=[1;\overline{1,2}]
  \end{align*}
\end{frame}

%%
\subsection{로렌즈 끌개}

%
\begin{luacode*}
  -- Differential equation of the Lorenz attractor
  function f(x,y,z)
     local sigma = 3
     local rho = 26.5
     local beta = 1
     return {sigma*(y-x), -x*z + rho*x - y, x*y - beta*z}
  end

  -- Code to write PGFplots data as coordinates
  function print_LorAttrWithEulerMethod(h,npoints,option)
     -- The usual starting point (x0,y0,z0)
     local x0 = 0.0
     local y0 = 1.0
     local z0 = 0.0
     -- we add a random number between -0.25 and 0.25
     local x = x0 + (math.random()-0.5)/2
     local y = y0 + (math.random()-0.5)/2
     local z = z0 + (math.random()-0.5)/2
     if option~=[[]] then
        tex.sprint("\\addplot3["..option.."] coordinates{")
     else
        tex.sprint("\\addplot3 coordinates{")
     end
     -- we dismiss the first 100 points to go into the attractor
     for i=1, 100 do
        m = f(x,y,z)
        x = x + h * m[1]
        y = y + h * m[2]
        z = z + h * m[3]
     end
     for i=1, npoints do
        m = f(x,y,z)
        x = x + h * m[1]
        y = y + h * m[2]
        z = z + h * m[3]
        tex.sprint("("..x..","..y..","..z..")")
     end
     tex.sprint("}")
  end
\end{luacode*}

\newcommand\addLUADEDplot[3][]{%
  \luadirect{print_LorAttrWithEulerMethod(#2,#3,[[#1]])}}

\begin{frame}{로렌즈 끌개}
  \begin{minipage}[c][1.0\height][c]{.6\textwidth}
    \pgfplotsset{width=\textwidth}
    \begin{tikzpicture}
      \begin{axis}
        \addLUADEDplot[color=red,smooth]{0.02}{1000};
        \addLUADEDplot[color=green,smooth]{0.02}{1000};
        \addLUADEDplot[color=blue,smooth]{0.02}{1000};
        \addLUADEDplot[color=cyan,smooth]{0.02}{1000};
        \addLUADEDplot[color=magenta,smooth]{0.02}{1000};
        \addLUADEDplot[color=yellow,smooth]{0.02}{1000};
      \end{axis}
    \end{tikzpicture}
  \end{minipage}%
  \begin{minipage}[c][1.0\height][c]{.4\textwidth}
    \small
    \begin{align*}
      x'(t) &= \sigma (y(t)-x(t)), \\
      y'(t) &= -x(t) z(t) + \rho x(t) - y(t), \\
      z'(t) &= x(t) y(t) - \beta z(t).
    \end{align*}
  \end{minipage}

  \begin{center}
    \href{http://www.unirioja.es/cu/jvarona/downloads/%
      numerical-methods-luatex.pdf}{Numerical methods with \hologo{LuaLaTeX}}
  \end{center}
\end{frame}

%%
\subsection{복면산 (覆面算; Alphametics)}

\newcolumntype{R}{>{\tt}r}

\newcommand{\tal}[3]{%
  \begin{array}{c@{\,}R}
    & #1 \\
    + & #2 \\
    \cmidrule(lr){1-2}
    & #3
  \end{array}}
%
\begin{frame}[fragile]{복면산 (覆面算; Alphametics)}
  \begin{itemize}
  \item 문자를 이용하여 표현된 수식에서 각 문자가 나타내는 숫자를 알아내는 문제
  \item 하나의 문자는 하나의 숫자를 나타내고, 첫 번째 자리의 숫자는 0이 아니다.
  \end{itemize}
  \medskip
  {\Large
  \[
  \def\arraystretch{0.85}
  \tal{SEND}{MORE}{MONEY}
  \quad
  \tal{BATMAN}{GOTHAM}{NIGHTS}
  \quad
  \tal{ZEROES}{ONES}{BINARY}
  \]
  
  \[ \verb!VIOLIN+VIOLIN+VIOLA=TRIO+SONATA! \]
  }
\end{frame}

% for alphametics
\luadirect{dofile(kpse.find_file(\luastring{alphapuzzle.lua}))}
\newcommand{\alphametics}[2][v]{%
  \luadirect{puzzle(\luastring{#1},\luastring{#2})}}

%
\begin{frame}{복면산}
  \vspace{-4mm}
  \large
  \def\arraystretch{0.85}
  \alphametics{INTO+ONTO+CANON+INTACT+AMMONIA+OMISSION+DIACRITIC%
    +STATISTICS+ASSOCIATION+ANTIMACASSAR+CONTORTIONIST+NONDISCRIMINATION%
    +CONTRADISTINCTION= MISADMINISTRATION}
\end{frame}

%
\begin{frame}[fragile]{복면산}
  \large
  \begin{verbatim}
  \alphametics{INTO+ONTO+CANON%
    +INTACT+AMMONIA+OMISSION%
    +DIACRITIC+STATISTICS%
    +ASSOCIATION+ANTIMACASSAR%
    +CONTORTIONIST%
    +NONDISCRIMINATION%
    +CONTRADISTINCTION%
    =MISADMINISTRATION}
\end{verbatim}
\end{frame}

%
\begin{frame}[fragile]{복면산}
\begin{verbatim}
\alphametics[h]{VIOLIN+VIOLIN+VIOLA=TRIO+SONATA}
\end{verbatim}%
\alphametics[h]{VIOLIN+VIOLIN+VIOLA=TRIO+SONATA}
\end{frame}

%
\begin{frame}{발표자료}
  \begin{itemize}
  \item \url{https://github.com/sjnam/2016-kts}
  \item \url{https://github.com/sjnam/luagmp}
  \end{itemize}
\end{frame}

%
\plain{\huge ¿Tienes alguna pregunta?}

\end{document}

