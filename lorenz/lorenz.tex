\documentclass{article}

\usepackage{url,microtype,upquote}
\usepackage{luacode}
\usepackage{pgfplots}
\pgfplotsset{compat=1.5} % to avoid warning

\begin{document}

The Lorenz attractor is a strange attractor that arises in a system of
equations describing the $2$-dimensional flow of a fluid of uniform depth,
with an imposed vertical temperature difference.
In the early 1960s, Lorenz discovered the chaotic behavior of a
simplified $3$-dimensional system of this problem,
now known as the Lorenz equations:
\[
\left\{
\begin{array}{l}
  x'(t) = \sigma (y(t)-x(t)), \\
  y'(t) = -x(t) z(t) + \rho x(t) - y(t), \\
  z'(t) = x(t) y(t) - \beta z(t).
\end{array}
\right.
\]
The parameters $\sigma$, $\rho$, and $\beta$ are usually assumed to be positive. Lorenz used the values $\sigma=10$, $\rho=28$ and $\beta=8/3$. The system exhibits a chaotic behavior for these values; in fact, it became the first example of a chaotic system.

Figure~\ref{fig:LA} shows the numerical solution of the Lorenz equations calculated with
$\sigma = 3$, $\rho = 26.5$ and $\beta = 1$.
Six orbits starting at several initial points close to $(0,1,0)$ are plotted in different colors; all of them converge to the $3$-dimensional chaotic attractor known as the Lorenz attractor.

\begin{figure}
\begin{luacode*}
  -- Differential equation of the Lorenz attractor
  function f(x,y,z)
     local sigma = 3
     local rho = 26.5
     local beta = 1
     return {sigma*(y-x), -x*z + rho*x - y, x*y - beta*z}
  end

  -- Code to write PGFplots data as coordinates
  function print_LorAttrWithEulerMethod(h,npoints,option)
     -- The usual starting point (x0,y0,z0)
     local x0 = 0.0
     local y0 = 1.0
     local z0 = 0.0
     -- we add a random number between -0.25 and 0.25
     local x = x0 + (math.random()-0.5)/2
     local y = y0 + (math.random()-0.5)/2
     local z = z0 + (math.random()-0.5)/2
     if option~=[[]] then
        tex.sprint("\\addplot3["..option.."] coordinates{")
     else
        tex.sprint("\\addplot3 coordinates{")
     end
     -- we dismiss the first 100 points to go into the attractor
     for i=1, 100 do
        m = f(x,y,z)
        x = x + h * m[1]
        y = y + h * m[2]
        z = z + h * m[3]
     end
     for i=1, npoints do
        m = f(x,y,z)
        x = x + h * m[1]
        y = y + h * m[2]
        z = z + h * m[3]
        tex.sprint("("..x..","..y..","..z..")")
     end
     tex.sprint("}")
  end
\end{luacode*}

\newcommand\addLUADEDplot[3][]{%
  \directlua{print_LorAttrWithEulerMethod(#2,#3,[[#1]])}%
}
\centering
\pgfplotsset{width=.8\hsize}
\begin{tikzpicture}
  \begin{axis}
    % SYNTAX: Solution of the Lorenz system
    % with step h=0.02 sampled at 1000 points.
    \addLUADEDplot[color=red,smooth]{0.02}{1000};
    \addLUADEDplot[color=green,smooth]{0.02}{1000};
    \addLUADEDplot[color=blue,smooth]{0.02}{1000};
    \addLUADEDplot[color=cyan,smooth]{0.02}{1000};
    \addLUADEDplot[color=magenta,smooth]{0.02}{1000};
    \addLUADEDplot[color=yellow,smooth]{0.02}{1000};
  \end{axis}
\end{tikzpicture}
\caption{The Lorentz attractor (six orbits starting at several initial points).\label{fig:LA}}
\end{figure}

\end{document}
